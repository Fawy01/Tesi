\cleardoublepage
\phantomsection
\pdfbookmark{Sommario}{Sommario}
\begingroup
\let\clearpage\relax
\let\cleardoublepage\relax
\let\cleardoublepage\relax

\chapter*{Sommario}

L'obiettivo del presente documento è mostrare il lavoro svolto durante il periodo di stage, dal laureando Fabio Pantaleo, presso l'azienda CWBI. \\
Lo scopo principale del progetto è analizzare uno dei rami del CRM\glsfirstoccur \;: il ticketing\glsfirstoccur. \\
Il primo passo per lo sviluppo del modulo web relativo al ticketing è l'analisi del problema con la conseguente raccolta dei requisiti primari in modo tale da elaborare i casi d'uso della nostra applicazione. Questa prima fase è molto importante per il ciclo di vita del nostro prodotto in quanto rappresenta la base di partenza per la costruzione del modello di dati.
Le prime funzionalità individuate sono la creazione, modifica ed eliminazione di un ticket da parte di un utente.  \\
Il prossimo passo è definire in che modo l'utente si interfaccia con le funzioni del modulo e quindi con quali componenti, anche visive, deve interagire per raggiungere lo scopo che si è prefissato.\\ 
Durante questo periodo è iniziata un ulteriore fase di analisi per introdurre nuove funzionalità, come il commento in tempi asincroni di un ticket da diverse utenti, che arricchiscono il modulo.
L'ultimo obiettivo è gestire il tipo di utente che utilizza il modulo per offrire diverse feature in base al livello di autorizzazione che un utente possiede.\\

\noindent L'IDE\glsfirstoccur \;utilizzato è Eclipse e il linguaggio per lo sviluppo del Model\glsfirstoccur \;è java, supportato da vari framework (struts2, maven, hibernate, Spring, JEE/ Spring, ecc...). Per il front-end sono utilizzati invece: HTML5, css3, Bootstrap, jsp. 

%\vfill

%\selectlanguage{english}
%\pdfbookmark{Abstract}{Abstract}
%\chapter*{Abstract}

%\selectlanguage{italian}

\endgroup

\vfill
