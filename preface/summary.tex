\cleardoublepage
\phantomsection
\pdfbookmark{Sommario}{Sommario}
\begingroup
\let\clearpage\relax
\let\cleardoublepage\relax
\let\cleardoublepage\relax

\chapter*{Sommario}

L'obiettivo del presente documento è mostrare il lavoro svolto durante il periodo di stage da parte del laureando Fabio Pantaleo presso l'azienda CWBI.\\
Lo scopo principale del progetto era l'analisi di uno dei rami del CRM\glsfirstoccur \;: il ticketing\glsfirstoccur . \\
Il primo passo per lo sviluppo del modulo web relativo al ticketing consisteva nell'analisi del problema, seguita dalla raccolta dei requisiti primari, al fine di elaborare i casi d'uso per la nostra applicazione. Questa fase iniziale era di grande importanza per il ciclo di vita del nostro prodotto, poiché rappresentava la base di partenza per la costruzione del modello di dati.
Le prime funzionalità individuate includevano la creazione, la modifica e l'eliminazione di un ticket da parte di un utente.\\
Il passo successivo è stato definire in che modo l'utente si sarebbe interfacciato con le funzioni del modulo e con quali componenti, anche visive, avrebbe dovuto interagire per raggiungere i propri obiettivi.\\
Durante quel periodo, è iniziata una fase di analisi aggiuntiva per introdurre nuove funzionalità, come il commento asincrono su un ticket da parte di diversi utenti, che arricchivano il modulo.
L'obiettivo finale era gestire il tipo di utente che utilizzava il modulo per offrire diverse funzionalità in base al livello di autorizzazione posseduto dall'utente.\\


%\vfill

%\selectlanguage{english}
%\pdfbookmark{Abstract}{Abstract}
%\chapter*{Abstract}

%\selectlanguage{italian}

\endgroup

\vfill
