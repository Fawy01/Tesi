\chapter{Conclusioni}
\label{cap:conclusioni}

\section{Consuntivo finale}
La pianificazione redatta ad inizio stage sul \textit{Piano di Lavoro} ha subito delle variazione delle variazione rispetto a quante ore sono state effettivamente dedicate per ogni attività.\\ La formazione iniziale e lo studio delle tecnologie utilizzate in azienda hanno richiesto più di 40 ore, come previsto. Infatti la comprensione della struttura aziendale è stata una delle parti cruciali per iniziare lo sviluppo del progetto. Quindi non solo sono stati affrontati e appresi le nuove tecnologie e framework già presenti, ma è stato fondamentale capire come questi elementi interagiscono tra di loro nell'architettura dell'azienda. \\
Un'altra attività che ha richiesto più di quanto previsto è stata lo sviluppo della soluzione ma non per la complessità di codifica del codice, bensì per la \textit{way of working} da intraprendere. Essenziale era avere le idee chiare e procedere per passi; non immaginare fin da subito l'intero prodotto con tutte le sue parti, ma lavorare su una singola componente per volta e svilupparne le caratteristiche. 
Per le attività rimanenti le ore previste sono state più che sufficienti e quelle non utilizzate sono state dedicate alle attività già citate sopra. \\
Le 300 ore totali previste sono state quindi rispettate grazie soprattutto agli strumenti messi a disposizione dall'azienda che facilitano il lavoro rendendolo veloce ed intuitivo.

\section{Raggiungimento degli obiettivi}
Gli obiettivi definiti con il tutor \textit{Roberto Martina} ad inizio stage sono stati ampiamente raggiunti. Lo studio e la comprensione della struttura aziendale sono stati eccellenti e cruciali per il raggiungimenti degli altri obiettivi, come lo sviluppo della webapp. \\
Tutte le componenti e funzionalità previste per il modulo sviluppato sono state integrate e testate per verificarne il corretto funzionamento, così come i requisiti obbligatori raccolti nella fase di "analisi dei requisiti" sono stati soddisfatti. 
\newpage
\section{Valutazione degli strumenti utilizzati}
L'\textit{IDE} utilizzato è Eclipse, già personalmente conosciuto in una versione molto precedente a quella utilizzata e riscoperto ancora più diretto e di facile utilizzo, con la possibilità di integrare nuove componentistiche per utilizzo professionale.\\
Per la visualizzazione dell'applicazione sul browser è stato utilizzato \textit{Tomcat} che si è rivelato un applicativo perfetto per lo sviluppo di applicazioni web. Alla creazione di un server \textit{Tomcat} c'era la possibilità di scegliere tra diverse opzioni in base alla versioni di Java e dei framework presenti. Configurabile facilmente con i file \textit{jar} essenziali per il corretto funzionamento del progetto.\\
In generale gli strumenti utilizzati sono stati adeguati per lo sviluppo di un'applicazione Java e li prenderò sicuramente in considerazione anche per progetti futuri.
 
\section{Miglioramenti e future estensioni}
Il modulo della webapp sviluppato, come detto precedentemente, ha soddisfatto tutti i presupposti elaborati e concordati per lo stage con il tutor. Anche il lato utente è stato ampiamente trattato, adattandolo alle linee guida dell'azienda per quanto riguarda il \textit{front-end}.\\
Tuttavia una crescita cruciale che dovrebbe affrontare il modulo riguarda il rispetto delle norme di accessibilità attuali. Infatti la maggior parte degli elementi di \textit{front-end} non sono allineati per quanto riguarda l'accessibilità, la quale negli ultimi anni è diventata la regola fondamentale che ogni applicazione web cerca di integrare.\\
Un ulteriore miglioramento deve essere l'ottimizzazione delle interazioni tra il modulo sviluppato e quelli già presenti. L' obiettivo punta al rifacimento di alcune delle classi presenti all'interno della \textit{webapp}, in modo da avere una più solida connessione con il nuovo modulo. \\
Durante la fase di sviluppo è stata ideata una nuova feature da integrare nel progetto ma lasciata in disparte per una futura estensione del modulo. L'idea era quella di fornire all'utente un' interfaccia di messaggistica per discutere con gli altri utenti sui diversi ticket aperti in tempo reale. L'introduzione di questa funzionalità richiedeva tuttavia un' ulteriore fase di analisi dei requisiti che avrebbe portato ad uno slittamento della data di fine progetto. 