\chapter{Conclusioni}
\label{cap:conclusioni}

\section{Consuntivo finale}
La pianificazione redatta ad inizio stage sul \textit{Piano di Lavoro} ha subito delle variazioni rispetto a quante ore sono state effettivamente dedicate per ogni attività.\\ La formazione iniziale e lo studio delle tecnologie utilizzate in azienda hanno richiesto più ore di quelle preventivate. Infatti la comprensione della struttura aziendale è stata una delle parti cruciali per iniziare lo sviluppo del progetto. Non solo sono state apprese le nuove tecnologie e i framework, ma è stato essenziale capire come questi elementi interagivano tra di loro nell'architettura dell'azienda. \\
Un'altra attività che ha richiesto più tempo è stata lo sviluppo della soluzione, non per la complessità di codifica del codice,  bensì per la \textit{way of working} da intraprendere. Un punto cruciale è stato avanzare meticolosamente; invece di immaginare immediatamente l'intero prodotto con tutte le sue parti, ci si è concentri su un singolo componente alla volta, sviluppandone gradualmente le caratteristiche.
Per le attività rimanenti le ore previste sono state più che sufficienti e quelle non utilizzate sono state dedicate alle attività già citate sopra. \\
Le 300 ore totali previste sono state quindi rispettate grazie e soprattutto agli strumenti messi a disposizione dall'azienda, che hanno facilitato il lavoro rendendolo veloce ed intuitivo.

\section{Raggiungimento degli obiettivi}
Gli obiettivi definiti con il tutor \textit{Roberto Martina} ad inizio stage sono stati ampiamente raggiunti. Lo studio e la comprensione della struttura aziendale sono stati indispensabili per il raggiungimento degli altri obiettivi, come lo sviluppo della webapp. \\
Tutte le componenti e funzionalità previste per il modulo sviluppato sono state integrate e testate per verificarne il corretto funzionamento, così come i requisiti obbligatori raccolti nella fase di "analisi dei requisiti" sono stati soddisfatti. 
\newpage
\section{Valutazione degli strumenti utilizzati}
L'\textit{IDE} utilizzato è Eclipse, che conoscevo personalmente da una versione precedente a quella utilizzata. Ho avuto modo di riscoprirlo e sperimentare quanto sia diretto e di facile utilizzo.\\
Per la visualizzazione dell'applicazione sul browser è stato utilizzato \textit{Tomcat} che si è rivelato un applicativo perfetto per lo sviluppo di applicazioni web. Durante la creazione di un server \textit{Tomcat} c'è stata la possibilità di scegliere tra diverse opzioni in base alla versioni di Java e dei framework presenti. Il server è risultato essere facilmente configurabile con i file \textit{jar} essenziali per il corretto funzionamento del progetto.\\
In generale gli strumenti utilizzati sono stati adeguati per lo sviluppo di un'applicazione Java e li prenderò sicuramente in considerazione anche per progetti futuri.
 
\section{Miglioramenti e future estensioni}
Il modulo sviluppato nella webapp , come detto precedentemente, ha soddisfatto tutti i presupposti elaborati e concordati per lo stage con il tutor. Anche il lato utente è stato ampiamente trattato, adattandolo alle linee guida dell'azienda per quanto riguarda il \textit{front-end}.\\
Tuttavia, una necessaria evoluzione del modulo dovrebbe riguardare l'aderenza alle attuali norme di accessibilità.  Attualmente, la maggior parte degli elementi del \textit{front-end} non è allineata con tali normative, le quali negli ultimi anni sono diventate una regola imprescindibile che ogni applicazione web cerca di implementare.\\
Un ulteriore miglioramento deve essere l'ottimizzazione delle interazioni tra il modulo sviluppato e quelli già presenti. L' obiettivo punta al rifacimento di alcune delle classi presenti all'interno della \textit{webapp}, in modo da avere una più solida connessione con il nuovo modulo. \\
Durante la fase di sviluppo è stata ideata una nuova feature da integrare nel progetto ma lasciata in disparte per una futura estensione del modulo. L'idea era quella di fornire all'utente un'interfaccia di messaggistica per discutere in tempo reale con gli altri utenti sui ticket aperti. L'introduzione di questa funzionalità richiedeva però un' ulteriore fase di analisi dei requisiti, che avrebbe portato ad uno slittamento della data di fine progetto. \\
In conclusione, il modulo sviluppato costituisce una solida base in cui integrare ulteriori miglioramenti e innovazioni, garantendo che la webapp rimanga sempre allineata alle esigenze dell'azienda.
\clearpage
\mbox{}
\newpage