\chapter{Conclusioni}
\label{cap:conclusioni}

\section{Consuntivo finale}
La pianificazione redatta ad inizio stage sul \textit{Piano di Lavoro} ha subito delle variazione delle variazione rispetto a quante ore sono state effettivamente dedicate per ogni attività.\\ La formazione iniziale e lo studio delle tecnologie utilizzate in azienda hanno richiesto più di 40 ore, come previsto. Infatti la comprensione della struttura aziendale è stata una delle parti cruciali per iniziare lo sviluppo del progetto. Quindi non solo sono stati affrontati e appresi le nuove tecnologie e framework già presenti, ma è stato fondamentale capire come questi elementi interagiscono tra di loro nell'architettura dell'azienda. \\
Un'altra attività che ha richiesto più di quanto previsto è stata lo sviluppo della soluzione ma non per la complessità di codifica del codice, bensì per la \textit{way of working} da intraprendere. Essenziale è avere le idee chiare e procedere per passi; non immaginare fin da subito l'intero prodotto con tutte le sue parti, ma lavorare su una singola componente per volta e svilupparne le caratteristiche. 
Per le attività rimanenti le ore previste sono state più che sufficienti e quelle non utilizzate sono state dedicate alle attività già citate sopra. \\
Le 300 ore totali previste sono state quindi rispettate grazie soprattutto agli strumenti messi a disposizione dall'azienda che facilitano il lavoro rendendolo veloce ed intuitivo.

\section{Raggiungimento degli obiettivi}
Gli obiettivi definiti con il tutor \textit{Roberto Martina} ad inizio stage sono stati ampiamente raggiunti. Lo studio e la comprensione della struttura aziendale sono stati eccellenti e cruciali per il raggiungimenti degli altri obiettivi, come lo sviluppo della webapp. \\
Tutte le componenti e funzionalità previste per il modulo sviluppato sono state integrate e testate per verificarne il corretto funzionamento, così come i requisiti obbligatori raccolti nella fase di "analisi dei requisiti" sono stati soddisfatti. 
\section{Valutazione degli strumenti utilizzati}
L'\textit{IDE} utilizzato è Eclipse, già personalmente conosciuto in una versione molto precedente a quella utilizzata e riscoperto ancora più diretto, con la possibilità di integrare nuove componentistiche per utilizzo professionale.\\
Per la visualizzazione dell'applicazione sul browser è stato utilizzato \textit{Tomcat} che si è rivelato un applicativo perfetto per lo sviluppo di applicazioni web. Alla creazione di un server \textit{Tomcat} c'è la possibilità di scegliere tra diverse opzioni in base alla versioni di Java e dei framework presenti. Configurabile facilmente con i file \textit{jar} essenziali per il corretto funzionamento del progetto.
In generale gli strumenti utilizzati sono adeguati per lo sviluppo di un'applicazione Java e personalmente presi in considerazione negli sviluppi futuri.
 
\section{Valutazione personale}
