
\chapter{Glossario}
\label{cap:glossario}

\section*{A}
\begin{itemize}
\item \textbf{Ajax}: tecnica per lo sviluppo di pagine dinamiche che non richiedono la ricarica della pagina.
\end{itemize}

\section*{B}
\begin{itemize}
\item \textbf{Baseapp}: applicazione sviluppata dall'azienda CWBI.
\item \textbf{Boostrap}: framework utilizzato per il design delle pagine web.
\end{itemize}

\section*{C}
\begin{itemize}
\item \textbf{CRM}: sistema utilizzato dalle azienda per gestire i rapporti con i clienti.
\item \textbf{CRUD}: acronimo che racchiude le quattro operazione principali di un'applicazione: \textit{create}, \textit{read},\textit{update},\textit{delete}.
\item \textbf{Css}: linguaggio usato per gestire il design e la presentazione delle pagine web.
\item \textbf{CWBI}: acronimo di \textbf{Codice Web Banking Innovation}.
\item \textbf{CWGEST}: applicazione sviluppata dall'azienda CWBI.
\end{itemize}

\section*{D}
\begin{itemize}
\item \textbf{DAO}: acronimo di \textbf{Data Access Object, è un design pattern per lo sviluppo di applicazioni.}
\end{itemize}

\section*{H}
\begin{itemize}
\item \textbf{Hibernate}: framework che gestisce il rapporto tra database e applicazione Java.
\item \textbf{HTML}: Acronimo di HyperText Markup Language permette di immaginare e formattare pagine
collegate fra di loro attraverso link.
\end{itemize}


\section*{J}
\begin{itemize}
\item \textbf{JSP}: acronimo di JavaServer Page.
\item \textbf{JSTL}: acronimo di JSP Standard Tag Library, è un'estensione di JSP.
\end{itemize}

\section*{M}
\begin{itemize}
\item \textbf{MVC}: pattern architetturale per lo sviluppo di un'applicazione. 
\end{itemize}

\section*{R}
\begin{itemize}
\item \textbf{Refactoring}: processo che prevede l'ottimizzazione del codice di un software. 
\item \textbf{Repository}: è una memoria in cui vengono memorizzati i file del sistema di versionamento.

\end{itemize}


\section*{S}
\begin{itemize}
\item \textbf{Sistema di versionamento}: è un sistema per tener traccia delle modifica di un file o software.
\item \textbf{Struts}: framework utilizzato per la comunicazione tra front-end e back-end.
\item \textbf{SVNKit}: sistema di versionamento di Java.

\end{itemize}

\section*{T}
\begin{itemize}
\item \textbf{Taconite}: framework utilizzato per le chiamate Ajax.
\item \textbf{Ticketing}: sistema per gestire ticket.
\end{itemize}

\section*{U}
\begin{itemize}
\item \textbf{UML}: acronimo di \textbf{Unified Modeling Language}, utilizzato per rappresentare l'architettura di un software.
\end{itemize}


\section*{X}
\begin{itemize}
\item \textbf{XML}: linguaggio di markup per la rappresentazione di dati strutturati.
\end{itemize}