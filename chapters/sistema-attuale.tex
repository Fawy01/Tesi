\chapter{Descrizione dello stage}

\section{Sistema attuale}
CWBI ha sviluppato CWGEST\glsfirstoccur , un'applicazione usata internamente all'azienda per la gestione, l'organizzazione e il tracciamento delle interazioni con utenti esterni, clienti e non, che supporta il personale offrendo una \textit{way of working}. \\
L'applicazione è divisa in due menu:
\begin{itemize}
	\item \textbf{Amministrativo}
	\item \textbf{Gestionale}
\end{itemize}
Ogni sezione ha all'interno diversi moduli, rispettivamente:\\
\textbf{Amministrativo}
\begin{itemize}
\item \textit{Administration Module};
\item \textit{User Registration Module};
\item \textit{User Menu Module};
\item \textit{Tracking Module}.
\end{itemize}
I moduli presenti in questo menu servono per la gestione di \textit{CWGEST} e offrono diverse funzionalità, come ad esempio la registrazione di nuovi utenti per accedere all'applicazione. Tutti questi moduli sono riservati all'utente amministratore e quindi non visibili all' utente generico. \\
\textbf{Gestionale}
\begin{itemize}
\item Ticket 
\item Offerta
\item Consuntivazione
\item Progetto Cliente
\end{itemize}
Alcuni dei moduli di quest' ultimo menu non sono attivi oppure c'è il bisogno, da parte dell'azienda, di eseguire un'operazione di \textit{refactoring\glsfirstoccur}\; su quelli attualmente in funzione, con l'obiettivo di estendere l'utilizzo dell'applicazione ad agenti esterni come, ad esempio, un cliente. 

