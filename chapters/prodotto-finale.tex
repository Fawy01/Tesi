\chapter{Prodotto finale}
\label{cap:prodotto finale}

\section{Pagina iniziale}
Una volta effettuato il \textit{login} l'utente poteva scegliere il modulo a cui voleva accedere della webapp. L'utente, accedendo al modulo Ticket, veniva indirizzato in una pagina che era composta da due sezioni:
\begin{itemize}
\item \textbf{Home Ticet};
\item \textbf{Menu Ticket}.
\end{itemize}

\subsection{Home Ticket}
La prima pagina visualizzata era la pagina di \textbf{Home Ticket}. In questa pagina erano presenti tre liste, ognuna con una modalità di visualizzazione di ticket diversa:
\begin{itemize}
\item \textbf{Aperti da me}: visualizzava tutti i ticket aperti dall'utente che stava accedendo la pagina;
\item \textbf{Assegnati a me}: visualizzava tutti i ticket assegnati all'utente che stava accedendo la pagina. Infatti un dipendente CWBI visualizzava i ticket assegnati a lui.
\item \textbf{Recenti}: visualizzava gli ultimi dieci ticket aperti, in generale.
\end{itemize}

\begin{figure}[H]
	\centering
    \includegraphics[width=1.1\columnwidth]{ticketApertidaMe1} 
    \includegraphics[width=1.1\columnwidth]{ticketApertidaMe2} 
    \caption{Home Ticket}
\end{figure}

\subsection{Menu Ticket}
La seconda sezione disponibile all'entrata nel modulo Ticket era la pagina che mostra il menù, in cui si poteva scegliere di aprire un nuovo ticket oppure di effettuare una ricerca.
 
\begin{figure}[H]
	\centering
    \includegraphics[width=0.7\columnwidth]{menu} 
    \caption{Menu Ticket}
\end{figure}

\newpage

\section{Nuovo Ticket}
La creazione di un nuovo ticket si divideva in due step:
\begin{enumerate}
\item Si doveva scegliere l'azienda che stava aprendo il nuovo ticket;
\item In base all'azienda scelta, venivano visualizzati i progetti disponibili su cui aprire il ticket; si compilavano gli altri campi per l'apertura.
\end{enumerate}

 
\begin{figure}[H]
\bigskip
	\centering
    \includegraphics[width=0.7\columnwidth]{NewTicket-Step1} 
    \caption{Nuovo Ticket - Step 1}
\end{figure}

\begin{figure}[H]
\bigskip
	\centering
	
    \includegraphics[width=1.5\columnwidth]{newTicket-Step2-prima parte} 
\end{figure}

\begin{figure}[H]
\bigskip
	\centering

        \includegraphics[width=1.5\columnwidth]{newTicket-Step2-seconda parte} 
    \caption{Nuovo Ticket - Step 2}
\end{figure}

\bigskip
\section{Ricerca Ticket}
La pagina di ricerca Ticket offriva un spazio in cui erano visualizzati tutti i ticket secondo i criteri di ricerca. I filtri selezionabili si trovavano sul menu a sinistra della pagina e per effettuare la ricerca bastava premere sul pulsante "Cerca". Per ogni ticket presente nella lista erano presenti i campi delle caratteristiche che lo rappresentavano. Inoltre erano disponibili i pulsanti di modifica e di dettaglio.

\begin{figure}[H]
\bigskip
	\centering
    \includegraphics[width=0.35\columnwidth]{ricercaTicket1.1} 
        \includegraphics[width=0.35\columnwidth]{ricercaTicket1.2} 
\end{figure}

\begin{figure}[H]
\bigskip
     \includegraphics[width=1.3\columnwidth]{ricercaTicket2} 
    \caption{Ricerca Ticket}
\end{figure}

\bigskip
\section{Dettaglio Ticket}
Il dettaglio di un ticket era visualizzato attraverso il pulsante di "Dettaglio" presente in ogni riga delle liste di ticket. La pagina di dettaglio, come dice il nome, visualizzava le caratteristiche del ticket in modo dettagliato. \\ 
La parte iniziale della pagina era composta da:\
\medskip
\\textbf{Parte sinistra}
\begin{itemize}
\item \textit{Titolo del Ticket};
\item \textit{Stato};
\item \textit{Priorità};
\item \textit{Tipo};
\item \textit{Ambiente}.
\end{itemize}

\medskip
\noindent
\textbf{Parte destra}
\begin{itemize}
\item \textit{Download dell'allegato};
\item \textit{Chiudi/Apri};
\item \textit{Modifica};
\item \textit{Elimina};
\end{itemize}
\bigskip
\bigskip
\begin{figure}[H]
	\centering
	    \includegraphics[width=1.1\columnwidth]{dettaglioTicket1} 
     \includegraphics[width=0.7\columnwidth]{dettaglioTicket2} 
    \caption{Parte iniziale Dettaglio Ticket}
\end{figure}


\noindent
La parte centrale era composta dalle rimanenti caratteristiche del ticket.\\
\bigskip
\begin{figure}[H]
	\centering
    \includegraphics[width=1.25\columnwidth]{dettaglioTicket3} 
\caption{Parte Centrale Dettaglio Ticket}
\end{figure}

\noindent
A fine pagina era posizionata la sezione dei commenti in cui erano presenti tutti i commenti inseriti dagli utenti, con la possibilità di pubblicare un nuovo commento. 

\begin{figure}[H]
	\centering
    \includegraphics[width=1.2\columnwidth]{dettaglioTicket4} 
\caption{Commenti Dettaglio Ticket}
\end{figure}
\newpage
\section{Commento Ticket}
Questa pagina serviva per lasciare una nota per un ticket.


\begin{figure}[H]
	\centering
    \includegraphics[width=0.8\columnwidth]{nota} 
\caption{Commenti Dettaglio Ticket}
\end{figure}
