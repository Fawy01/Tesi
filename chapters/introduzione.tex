\chapter{Introduzione}
\label{cap:introduzione}

\section{L'azienda}

CWBI\gls è una società italiana di sviluppo software specializzata nella fornitura di soluzioni internet e mobile per banche, assicurazioni e industria. Opera nel mercato dell' \textbf{Information Communication Technology} e fornisce ai propri clienti un supporto nello studio dei \textit{modelli business} e nella progettazione e realizzazione di software orientati alle ultime tecnologie in questo campo.\\
CWBI offre una vasta gamma di servizi quali:
\begin{itemize}
\item Sviluppo applicazioni e portali web-based
\item Sviluppo applicazioni mobile
\item Analisi e definizione dei processi organizzativi
\item Studi di navigabilità e usabilità
\end{itemize}  
\begin{figure}[!h]
     \centering
     \includegraphics[height= 1.5cm]{cwbi}
     \caption{CWBI}
\end{figure}

\section{Lo stage}
\subsection{Bisogni dell'azienda}
L'azienda ha sviluppato un'applicazione web per la gestione del personale e della clientela. La webapp è divisa in diversi moduli e solo alcuni di questi erano stati integrati e perfezionati, per essere utilizzati attivamente dal personale. Gli altri moduli presenti erano stati introdotti nell'applicazione ma non sviluppati in quanto non essenziali nel breve periodo. \\
Lo stage ha presentato l'introduzione di un nuovo modulo: \textbf{\textit{Ticket}}. Questa nuova feature è stata richiesta dall'azienda per gestire le segnalazioni dei propri clienti. Infatti, prima dello sviluppo del modulo \textit{Ticket}, l'azienda si interfacciava con le problematiche riscontrate dagli utenti utilizzando degli strumenti sì utili, ma non adatti allo scopo. Ad esempio, per tener traccia di una segnalazione, veniva utilizzato un documento Word, che non forniva dettagli utili per capire la vera natura del problema; inoltre i file potevano essere persi o eliminati da un momento all'altro. \\
CWBI ha quindi richiesto lo sviluppo del nuovo modulo per gestire al meglio le interazioni con i propri clienti, offrendo a quest'ultimi la possibilità di creare un nuovo ticket in qualsiasi momento.
\subsection{Il progetto}
Il progetto consisteva nella realizzazione del modulo \textit{Ticket}, integrandone tutte le funzionalità richieste dal tutor. \\
Prima di iniziare il progetto c'è stata una fase di studio dell'architettura dell'azienda, per comprendere come le entità presenti interagissero tra di loro. Una volta conclusa questa fase, è avvenuta l'analisi dei requisiti del progetto per individuare tutte le funzionalità che il modulo doveva mettere a disposizione. Dopo si sono sviluppate le classi utili al modulo \textit{Ticket}, prestando attenzione ad integrarle nel miglior modo possibile con le altre classi già presenti. \\
Una volta conclusa la parte di back-end, è stata sviluppato anche il front-end. Anche in questo scenario, è stato utili osservare la grafica degli altri moduli già ampiamente sviluppati dal personale CWBI, per rispettare appunto gli standard grafici dell'azienda. 


\section{Tecnologie utilizzate}
Nello sviluppo dei propri prodotti, CWBI si occupa sia della parte di \textit{back-end\glsfirstoccur\;} sia della parte di \textit{front-end\glsfirstoccur\;}.
Per la prima, è utilizzato Java\glsfirstoccur come linguaggio di programmazione, supportato dai vari \textit{framework\glsfirstoccur}; mentre per la parte destinata alla vista del cliente, sono utilizzati:
\begin{itemize}
\item HTML5\glsfirstoccur ;
\item Css\glsfirstoccur ;
\item Boostrap3/5\glsfirstoccur ;
\item JSP\glsfirstoccur .
\end{itemize}
Affiancata anche questa da \textit{framework} come: 
\begin{itemize}
\item JSTL\glsfirstoccur ;
\item Struts2\glsfirstoccur ;
\item Taconite\glsfirstoccur .
\end{itemize}
Per tracciare gli interventi relativi al codice, l'azienda si avvale di un sistema di \textit{versionamento\glsfirstoccur}\; con una \textit{repository\glsfirstoccur}\; in remoto, accessibile grazie a un \textit{toolkit} di Java: SVNKit\glsfirstoccur.

\section{Organizzazione del testo}
Riguardo la stesura del testo, relativamente al documento sono state adottate le seguenti convenzioni tipografiche:
\begin{itemize}
	\item gli acronimi, le abbreviazioni e i termini ambigui o di uso non comune menzionati vengono definiti nel glossario, situato alla fine del presente documento;
	\item per la prima occorrenza dei termini riportati nel glossario viene utilizzata la seguente nomenclatura: \emph{parola}\glsfirstoccur ;
	\item i termini in lingua straniera o facenti parti del gergo tecnico sono evidenziati con il carattere \emph{corsivo}.
\end{itemize}

\section{Struttura}
Il testo sarà composto dai seguenti capitoli:
\begin{itemize}
\item Introduzione
\item Descrizione del sistema attuale
\item Descrizione dello stage
\item Analisi dei requisiti
\item Progettazione e codifica
\item Verifica e validazione
\item Prodotto finale
\item Conclusioni
\end{itemize}