\chapter{Introduzione}
\label{cap:introduzione}

\section{L'azienda}

CWBI\gls è una società italiana di sviluppo software specializzata nella fornitura di soluzioni internet e mobile per banche, assicurazioni e industria. Opera nel mercato dell' \textbf{Information Communication Technology} e fornisce ai propri clienti un supporto nello studio dei \textit{modelli business} e nella progettazione e realizzazione di software orientati alle ultime tecnologie in questo campo.\\
CWBI offre una vasta gamma di servizi quali:
\begin{itemize}
\item Sviluppo applicazioni e portali web-based
\item Sviluppo applicazioni mobile
\item Analisi e definizione dei processi organizzativi
\item Studi di navigabilità e usabilità
\end{itemize}  
\begin{figure}[!h]
     \centering
     \includegraphics[height= 1.5cm]{cwbi}
     \caption{CWBI}
\end{figure}


\section{Tecnologie utilizzate}
Nello sviluppo dei propri prodotti, CWBI si occupa sia della parte di \textit{back-end\glsfirstoccur\;} sia della parte di \textit{front-end\glsfirstoccur\;}.
Per la prima, è utilizzato Java\glsfirstoccur come linguaggio di programmazione, supportato dai vari \textit{framework\glsfirstoccur}; mentre per la parte destinata alla vista del cliente, sono utilizzati:
\begin{itemize}
\item HTML5\glsfirstoccur ;
\item Css\glsfirstoccur ;
\item Boostrap3/5\glsfirstoccur ;
\item JSP\glsfirstoccur .
\end{itemize}
Affiancata anche questa da \textit{framework} come: 
\begin{itemize}
\item JSTL\glsfirstoccur ;
\item Struts2\glsfirstoccur ;
\item Taconite\glsfirstoccur .
\end{itemize}
Per tracciare gli interventi relativi al codice, l'azienda si avvale di un sistema di \textit{versionamento\glsfirstoccur}\; con una \textit{repository\glsfirstoccur}\; in remoto, accessibile grazie a un \textit{toolkit} di Java: SVNKit\glsfirstoccur.

\section{Organizzazione del testo}
Riguardo la stesura del testo, relativamente al documento sono state adottate le seguenti convenzioni tipografiche:
\begin{itemize}
	\item gli acronimi, le abbreviazioni e i termini ambigui o di uso non comune menzionati vengono definiti nel glossario, situato alla fine del presente documento;
	\item per la prima occorrenza dei termini riportati nel glossario viene utilizzata la seguente nomenclatura: \emph{parola}\glsfirstoccur ;
	\item i termini in lingua straniera o facenti parti del gergo tecnico sono evidenziati con il carattere \emph{corsivo}.
\end{itemize}

\section{Struttura}
Il testo sarà composto dai seguenti capitoli:
\begin{itemize}
\item Introduzione
\item Descrizione del sistema attuale
\item Descrizione dello stage
\item Analisi dei requisiti
\item Progettazione e codifica
\item Verifica e validazione
\item Conclusioni
\end{itemize}