\chapter{Verifica e validazione}
\label{cap:verifica-validazione}

Il seguente capitolo ha lo scopo di mostrare le tecniche di verifica e validazione del progetto, secondo le linee guida apprese durante il corso di Ingegneria del Software. \\
Verificare un prodotto ha l'obiettivo di controllare se l'introduzione di nuovi elementi nel codice ha generato dei problemi e se rispetta i requisiti designati. 
\\La validazione serve ad approvare il progetto qualora soddisfi tutti i requisiti imposti.

\section{Processo di Verifica}
Il processo di verifica è stato attuato durante tutto lo sviluppo del progetto per verificare le nuove funzionalità e comportamenti introdotti.\\
L'approccio all'introduzione di nuovi elementi con le relative funzioni è stato costantemente controllato da \textit{Roberto Martina}, responsabile di stage. Infatti  la tecnica più efficiente per lo sviluppo del codice è codificare una delle parti di webapp, verificare che il codice appena introdotto funzioni correttamente nel suo insieme e soltanto dopo collegarlo alle altri parti di codice già sviluppate.\\
Quindi, non si procede per codificare il "tutto" perché i requisiti potrebbero essere non soddisfatti e c'è il rischio di perdere l'obiettivo durante lo sviluppo; ma è rigoroso procedere per passi e per ognuno verificarne la correttezza.

\section{Processo di Validazione}
