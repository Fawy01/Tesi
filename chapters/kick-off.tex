\chapter{Descrizione dello stage}
\label{cap:descrizione-stage}

\section{Introduzione al progetto}
Visto i bisogni dell'azienda, l'obiettivo del modulo \textbf{Ticket} è quello di offrire un portale su cui gli utenti registrati, possono aprire, prendere in carico, assegnare ed eliminare dei ticket. \\
Il modulo è stato pensato sì per i dipendenti interni di CWBI, ma vuole offrire anche ai clienti un modo di segnalare in modo facile e veloce un qualsiasi tipo di problema sulle applicazioni utilizzate. Così anche per CWBI è semplice vedere chi e quando ha inviato una segnalazione, in modo da assegnare un dipendente per trovare una soluzione.\\ 
Inoltre per rendere più interattivo il gestionale ed avere un riscontro su quali operazioni sono state effettuate sul ticket, ogni utente potrà lasciare dei commenti in modo da far capire a chi prenderà in carico il ticket, a quale fase della soluzioni è arrivato.
\section{Analisi preventiva dei rischi}

Durante la fase di analisi iniziale sono stati individuati alcuni possibili rischi a cui si potrà andare incontro.
Si è quindi proceduto a elaborare delle possibili soluzioni per far fronte a tali rischi.\\

\begin{risk}{Performance del simulatore hardware}
    \riskdescription{le performance del simulatore hardware e la comunicazione con questo potrebbero risultare lenti o non abbastanza buoni da causare il fallimento dei test}
    \risksolution{coinvolgimento del responsabile a capo del progetto relativo il simulatore hardware}
    \label{risk:hardware-simulator} 
\end{risk}

\section{Obiettivi del progetto}


\section{Pianificazione}
