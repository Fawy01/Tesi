\chapter{Descrizione dello stage}
\label{cap:descrizione-stage}

\section{Introduzione al progetto}
Visto i bisogni dell'azienda, l'obiettivo del modulo \textbf{Ticket} è quello di offrire un portale su cui gli utenti registrati, possono aprire, prendere in carico, assegnare ed eliminare dei ticket. \\
Il modulo è stato pensato sì per i dipendenti interni di CWBI, ma vuole offrire anche ai clienti un modo di segnalare in modo facile e veloce un qualsiasi tipo di problema sulle applicazioni utilizzate. Così anche per CWBI è semplice vedere chi e quando ha inviato una segnalazione, in modo da assegnare un dipendente per trovare una soluzione.\\ 
Inoltre per rendere più interattivo il gestionale ed avere un riscontro su quali operazioni sono state effettuate sul ticket, ogni utente potrà lasciare dei commenti in modo da far capire a chi prenderà in carico il ticket, a quale fase della soluzioni è arrivato.\\
Lavorando su un'applicazione già esistente, una fase molto importante che il progetto prevede sono le attività di refactoring di moduli preesistenti affinché si adattino al nuovo modulo Ticket. 

\section{Obiettivi}
Lo stage definisce delle tappe fondamentali da raggiungere, sia per quanto riguarda lo sviluppo di un prodotto, ma soprattutto la formazione della persona è uno dei traguardi principali che l'azienda, il tutor aziendale e lo stagista hanno volontà di completare. Gli obiettivi quindi sono:
\begin{itemize}
\item la formazione del tirocinante affinché posso affrontare gli studi e il mondo del lavoro in un'ottica diversa, con nuove conoscenze e con la consapevolezza di un \textbf{metodo} di lavoro che aiuterà ad affrontare i futuri problemi attraverso pensiero logico e strategico;
\item lo sviluppo del modulo Ticket per supportare e facilitare i rapporti tra azienda e cliente.
\end{itemize}

\pagebreak

\section{Pianificazione}
Il lavoro si svolge nelle 300 ore obbligatorie per il tirocinio formativo e si suddivide in:
\begin{itemize}
\item Studio ed analisi dell'architettura già presente in azienda;
\item Raccolta dei requisiti del prodotto atteso;
\item Refactoring di codice di classi esistenti per adattarlo al prodotto;
\item Sviluppo del prodotto;
\item Test.
\end{itemize}

\noindent
Le ore si sono distribuite in 8 settimane lavorative, a loro volta suddivise nel particolare dedicando:
\begin{table}[h]%
\label{tab:tabella-pianificazione}
{\renewcommand{\arraystretch}{2}%}
\begin{tabularx}{\textwidth}{lXl}
\hline\hline
\textbf{Durata in ore} & \textbf{Attività}\\
\hline
40 & Formazione iniziale e introduzione tecnologie utilizzata JAVA/JEE lato server\\
\hline
40 & Formazione soluzione \textit{baseapp\glsfirstoccur} \; con apprendimento \textit{framework} di lavoro\\
\hline
68 & Analisi e raccolta requisiti progetto marketing\\
\hline
122 & Sviluppo soluzione (Realizzazione soluzione software in java \textit{back-end} e sviluppo \textit{front-end})\\
\hline
15 & Test e supporto UAT\\
\hline
15 & Documentazione progetto\\
\hline
\end{tabularx}
\smallskip
\caption{Tabella della pianificazione del lavoro}

\end{table}

