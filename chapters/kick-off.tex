\section{Introduzione al progetto}
Visto i bisogni dell'azienda, l'obiettivo del modulo \textbf{Ticket} era quello di offrire un portale su cui gli utenti registrati, potevano aprire, prendere in carico, assegnare ed eliminare un ticket. \\
Il modulo è stato pensato sì per i dipendenti interni di CWBI, ma voleva offrire anche ai clienti la possibilità di segnalare in modo facile e veloce un qualsiasi tipo di problema sulle applicazioni utilizzate. Così anche per CWBI sarebbe stato semplice vedere chi e quando ha inviato una segnalazione, in modo da assegnare un dipendente per trovare una soluzione.\\ 
Inoltre per rendere più interattivo il gestionale ed avere un riscontro su quali operazioni sono state effettuate sul ticket, ogni utente poteva commentare in modo da far capire a chi avrebbe preso in carico il ticket, a quale fase della soluzione si era arrivati.\\
Lavorando su un'applicazione già esistente, un'operazione cruciale che il progetto ha previsto sono le attività di refactoring di moduli preesistenti per adattarli al nuovo modulo \textit{Ticket}. 

\section{Obiettivi}
Lo stage ha definito delle tappe fondamentali da raggiungere, sia per quanto riguarda lo sviluppo di un prodotto, ma anche e soprattutto la formazione della persona è stato uno dei traguardi principali che l'azienda, il tutor aziendale e lo stagista avevano di completare. Gli obiettivi quindi erano:
\begin{itemize}
\item la formazione del tirocinante affinché possa affrontare gli studi e il mondo del lavoro in un'ottica diversa, con nuove conoscenze e con la consapevolezza di un \textbf{metodo} di lavoro che aiuterà ad affrontare i problemi futuri attraverso una \textit{way of working} solida;
\item lo sviluppo del modulo Ticket per supportare e facilitare i rapporti tra azienda e cliente.
\end{itemize}


\section{Pianificazione}
Il lavoro si è svolto nelle 300 ore obbligatorie per il tirocinio formativo e si suddivideva in:
\begin{itemize}
\item Studio ed analisi dell'architettura già presente in azienda;
\item Raccolta dei requisiti del prodotto atteso;
\item Refactoring di codice di classi esistenti per adattarlo al prodotto;
\item Sviluppo del prodotto;
\item Test.
\end{itemize}

\noindent
\newpage
Le ore si sono distribuite in 8 settimane lavorative, a loro volta suddivise nel particolare dedicando:
\begin{table}[H]%
\label{tab:tabella-pianificazione}
{\renewcommand{\arraystretch}{2}%}
\begin{tabularx}{\textwidth}{lXl}
\hline\hline
\textbf{Durata in ore} & \textbf{Attività}\\
\hline
40 & Formazione iniziale e introduzione tecnologie utilizzata JAVA/JEE lato server\\
\hline
40 & Formazione soluzione \textit{baseapp\glsfirstoccur} \; con apprendimento \textit{framework} di lavoro\\
\hline
68 & Analisi e raccolta requisiti progetto marketing\\
\hline
122 & Sviluppo soluzione (Realizzazione soluzione software in java \textit{back-end} e sviluppo \textit{front-end})\\
\hline
15 & Test e supporto UAT\\
\hline
15 & Documentazione progetto\\
\hline
\end{tabularx}
\smallskip
\caption{Tabella della pianificazione del lavoro}

\end{table}

